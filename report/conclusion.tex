This effort successfully showed the analysis of an eclipsing binary star system using photometric
data. We were able to create a distinct light curve that showed the system's basic eclipse by
cleaning and structuring a dataset of relative flux values. Using Python-based analysis tools,
significant eclipse parameters emerged from this, including as timing, depth, and length.

Using the inverse-square law and the given distance to the system, we were able to visualize the
relative flux and translate it into intrinsic luminosity. This highlighted the physical relevance
of the photometric observations and made it possible to evaluate the behavior of the system in a
deeper way.

The final annotated graphs confirmed the binary structure of the system and gave insight on the
eclipse's nature. All things considered, this effort proves the efficiency of combining
observational data with computational techniques to reveal astrophysical processes in a measurable
and repeatable manner.

To attempt to identify trends in system parameters, future research might compare several eclipsing
binaries or apply similar techniques to a larger dataset. In addition, more accurate measurements of
star masses, orbit characteristics, and inclination angles may be possible by fitting models to the
light curve.
