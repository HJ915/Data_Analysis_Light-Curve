The study of variable stars, and in particular eclipsing binaries, provides much-needed insights
into the properties and behavior of stellar systems. Eclipsing binaries, the system we observed,
are part of this category and are two stars that orbit each other and periodically eclipse.
This process causes measurable dips in the observable brightness of the system and can be
detected through photometric methods.

Photometry, the science of measuring light, allows astronomers to construct light curve graphs
showing observed flux over time of a system. These light curves play an essential role in
characterizing the physical properties of the celestial body in mind. Some of these physical
characteristics include orbital periods, size, shape, temperature difference, etc. Even without
the lack of high-resolution imaging, we can still learn quite a bit about far-reaching systems.

Recent studies have demonstrated the effectivness of photometric methods in analyzing 
eclipsing binary systems. For example, Mazeh, Tamuz, and North (2006) developed the 
Eclipsing Binary Automated Solved (EBAS), an automated algorithm that processes light curves.
EBAS allows efficient and fast interpretation of large datasets and shows how computational 
techniques can help our understanding of stellar systems \cite{mazeh2006automated}.

In this project, we applied photometric analysis techniques using Python code to a dataset. This
dataset contained flux measurements and the Julian Heliocentric time for each measurement of a
specified eclipsing binary. The goal was to analyze and identify various features of the system,
in particular the primary eclipse and the luminosity of the system. Meaning we need to use the flux
of the system over time to calculate luminosity, depth, and duration of the eclipse. We will then
apply the calculated information to a graph for visualization, showing how computational methods
can extract valuable information from large data sets.

