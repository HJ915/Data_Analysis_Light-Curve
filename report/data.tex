The data used in this project consists of time-series photometric observations recorded in the 
file \texttt{flux\_final.txt}. Each row is in the data set contains an Heliocentric Julian Date 
(HJD) value and the corresponding relative flux measurement of the observed binary star system.
The orginial raw data was obtained from a larger observational file and the processed to isolate
only the relevant columns: time and flux.

The data cleaning process was performed using Linux command-line tools to extract the third and 
fourth columans of the original dataset, which represented the HJD and \texttt{rel\_flux\_T!} 
values. A header row was added to label the columns for compatibility with Python's \texttt{pandas}
library. This simplified the data importing and analysis proves in the python script.

The head of the dataset is shown:

\begin{verbatim}
HJD_UTC    rel_flux_T1
2460634.648530    0.069839
2460634.649953    0.075013
2460634.651376    0.072118
\end{verbatim}

This format was primarly chosen to keep efficiency and ease with Python. The data is assumed to
be free of major outliers and errors. Some variability is shown but most likely due to the presence 
of natural atmospheric and eclipsing during collection. 
