This project showsa a detailed photometric analysis of an eclipsing binary system. We do this 
through the use of flux measurments across time. The main objective is to analyze the flux and 
Julian Heliocentric (HJD) measurments and find the primary eclipsing point for this curve.

In order to do this, I began with cleaning a set of data from my best observation night. Orginally
it was a .dat file with numerous headers and values, using linux commands I pulled the necessary
flux and HJD data. Once the data was "cleaned" I began writing the python script to read and
process the \texttt{flux\_final.txt} file containing the data. The script used standard scientific 
libraries to plot the light curve, calcualte an average, and detect the minimum flux value. The 
eclipse and duration were also calculated from the plotted data.

The inverse square law was applied to convert the relative flux into an estimate of the system's
intrinsic luminosity. After manually calculating this system distance from earth start at 
94.54 milliarcseconds, I found it was 34.13 light-years away. Changing the distance in lighyear value 
will be needed for various systems. 

The final output gives us three figures: a raw light curve, intrinsic luminosity curve, and a plot
highlighting the eclipses various characteristics. This method helps demonstrate how computational
methods can be used in the field of astrophysics to efficiently read through data. This code can 
take in and process any set of varying flux systems to find its characteriscs, speeding up 
the system intepretation process.  
